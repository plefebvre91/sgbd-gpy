\documentclass{article}
\usepackage[normalem]{ulem}

\usepackage[francais]{babel}
\usepackage[utf8]{inputenc} % \usepackage[latin1]{inputenc} L'option dépend du système sur lequel on compile
\usepackage[T1]{fontenc}

\newenvironment{mld}
{\par\begin{minipage}{\linewidth}\begin{tabular}{rp{\linewidth}}}
{\end{tabular}\end{minipage}\par}
\newcommand{\relat}[1]{\textsc{#1}}
\newcommand{\attr}[1]{\emph{#1}}
\newcommand{\prim}[1]{\uline{#1}}
\newcommand{\foreign}[1]{\#\textsl{#1}}

\title{Modèle relationnel en troisième forme normale}

\begin{document}
\maketitle{}

\begin{mld}
  \relat{Joueur} & (\prim{pseudo}, \attr{nom}, \attr{prenom}, \attr{mail}, \foreign{idCategorie}, \foreign{idPlateforme})\\
  \relat{Pouce} & (\prim{idPouce}, \attr{valeur}, \foreign{idJoueur}, \foreign{idCommentaire})\\
  \relat{Commentaire} & (\prim{\foreign{idJoueur}, \foreign{idJeu}}, \attr{note}, \attr{commentaire}, \attr{date}, \foreign{idPlateforme})\\
  \relat{Categorie} & (\prim{idCategorie}, \attr{nomCategorie})\\
  \relat{Jeu} & (\prim{idJeu}, \attr{nomJeu}, \foreign{idEditeur})\\
  \relat{Plateforme} & (\prim{idPlateforme}, \attr{nomPlateforme})\\
  \relat{Editeur} & (\prim{idEditeur}, \attr{nomEditeur})\\
  \relat{Appartient} & (\prim{\foreign{idCategorie}, \foreign{idJeu}})\\
  \relat{Est disponible} & (\prim{\foreign{idPlateforme}, \foreign{idJeu}})\\
\end{mld}

\end{document}
